\chapter{Literature Review}

\section{Background}
Improving the USDOD\footnote{United States Department of Defense} guarantee of  12.8 meter horizontal accuracy from the Global Positioning System Standard Positioning Service requires that positions calculated by a GPS receiver be augmented to correct for delays induced in the SV\footnote{Space vehicle} signal's travel through the ionosphere and troposphere~\citep{2001DoDGPSperf}. Correctors transmitted to and processed by a capable receiver are able to compensate for a variety of SV signal transmission delays and instrument errors to improve the position determined at the receiver's antenna. Correctors are derived from the difference in position calculated at a stationary reference receiver antenna and the actual location of the stationary antenna. The reference receiver determines the signal error for each SV in view of the antenna. The position differential is the product of all ``signals in space'' errors induced in the signal. SV signal errors accumulate from an orbit irregularities (i.e.\ gravitational effects, solar wind, or outdated ephemerides); satellite and receiver clock errors; ionospheric and tropospheric delay; and other identifiable factors~\citep{2004leick}. The reference and mobile receivers must receive the same SV signals for correctors to have an effect on the mobile position accuracy~\citep{2008USDoT_NDGPS}.

Current federal government augmentation systems, the Wide Area Augmentation Systems (WAAS) sponsored by the Federal Aviation Administration and the National Differential GPS (NDGPS) sponsored by the US Coast Guard, provide civilian users with mapping-grade\footnote{Defined here and generally accepted as 1 to 3 meter horizontal accuracy} position accuracy. The USDOT\footnote{United States Department of Transportation} was given presendential authority to develop and promote the use of civilian GPS augmentation systems.%USDOT authority for GPS in transportation applications
Presidential Decision Directive National Science and Technology Council (NSTC-6), designating USDOT to serve as the lead agency within the U.S. Government for all Federal civilian GPS matters. NSTC-6 commissioned the USDOT to:
\begin{quotation}
``Develop and implement U.S. Government augmentations to the basic GPS for transportation applications.
\begin{itemize}
\firmlist
	\item In cooperation with the Departments of Commerce, Defense and State, take the lead in promoting commercial applications of GPS technologies and the acceptance of GPS and U.S. Government augmentations as standards in domestic and international transportation systems.
	\item In cooperation with other departments and agencies, coordinate U.S. Government-provided GPS civil augmentation systems to minimize cost and duplication of effort~\citep{1996NSTC-6}.''
\end{itemize}
\end{quotation}

Federal government provided GPS\footnote{Non-GPS augmentation to GNSS systems is not provided by federal government systems} signal augmentation can be categorized by the augmenting signal transmitter location into Space Based Augmentation Systems (SBAS) and Ground Based Augmentation Systems (GBAS). SBAS use geosynchronous satellites to relay corrections from ground reference stations to the user, while GBAS send corrections from ground reference stations directly to the user.

\subsection{Space-Based Augmentation Systems}
Government sponsored and privately funded SBAS are available to commercial users. FAA sponsors WAAS for aviation users and consists of an integrity reference monitoring network, processing facilities, geostationary satellites, and control facilities. The central data processing sites generate navigation messages for the geostationary satellites and WAAS messages. The information is modulated on the GPS-like signal and broadcast to the users from geostationary satellites. WAAS corrections result in actual 95\% horizontal accuracy ranging from 0.481 to 1.521 meters across the Continental United States (CONUS)~\citep{WAAS09}.

WAAS is limited to broadcasting differential corrections for GPS SVs only. As with SV signals, the reception of correctors broadcast from an SBAS can be adversely affected by foliage, terrain, and building shadowing along the signal path from the SBAS SV to the user. The USDOT cites signal shadowing effects from a single geostationary point source as an objectionable characteristic for the use in railroad applications. This characteristic as a primary objection by the FRA to the use of an SBAS as part of an LDS~\citep{2008USDoT_NDGPS}.

Commercial SBAS subscription services enable horizontal accuracies to 6 cm @95\%~\citep{2005fugro}and are used primarily in precision agriculture applications which, due to their use in open fields, are relatively unaffected by loss of the correction signal on the north side of tree lines or terrain, and under heavy foliage cover.

\subsection{Ground-Based Augmentation Systems}
The National Differential GPS (NDGPS) is a GBAS that uses terrestrial Low Frequency (LF) radio in the 285-325 kHz band for transmission of correctors to NDGPS capable receivers. A desirable aspect of long wavelength (1052-922 m) LF radio is ground wave propagation. LF digital signals are favored by the USDOT for communicating correctors due to signal reception at distances up to 250 miles distant from a terrestrial reference station transmitter and LF.

The accuracy of NDGPS augmentation degrades at a rate of $\pm$~6.6 parts per million (ppm) distant from the reference receiver~\citep{2000FRA_gps_ant}. A USDOT report recognizes other problems in addition to the low data rate of the NDGPS signal
\begin{quotation}
``...is further degraded by computational and other uncertainties in user equipment and the ability of user equipment to compensate for other error sources such as multi-path interference and propagation distortions''~\citep{2008USDoT_NDGPS}.
\end{quotation} 

Even with these considerations, the USDOT selected NDGPS as the GBAS for transportation applications. The USDOT promotes NDGPS as the augmentation system of choice for enabling positive train control location determination systems. The FRA qualifies its support for NDGPS use in PTC by understanding the need for ``other supplemental techniques'' to meet the high degree of confidence required of an LDS~\citep{1995FRADiffe}. The USDOT \emph{2008 NDGPS Assessment Final Report} states that ``NDGPS with its current level of accuracy has not proven adequate for safety-level track separation information~\citep{2008USDoT_NDGPS}.''

\section{Absolute Track Location Measurement Systems}
% Theory and research specific to the topic
% \section{Theory/research specific to the topic}
The ``other supplemental techniques'' referenced in the USDOT NDGPS Assessment are reflected in patents that integrate differential GPS, inertial systems, and wheel mounted tachometers to produce optimal estimators for determining locomotive track occupancy~\citep{2007lockheed}. Supplemental techniques were demonstrated across a wide area of mainline track in an asset mapping system demonstrated by Allen, Mason, and Stevens.

Allen, Mason, and Stevens developed a rail borne track-mapping system as a cost saving alternative to remote sensing from an aerial platform. Their survey platform consisted of a HiRail vehicle equipped to utilized publicly available real time correctors from the NDGPS in addition to post processed observations from a cooperative\footnote{http://www.ngs.noaa.gov/CORS/Coop/} Continuously Operating Reference Station (CORS). The GPS instrument was augmented with tachometer and inertial measurement unit (IMU) inputs. The IMU was tightly coupled with the GPS. Allen reported that an initial calibration on a dedicated survey vehicle took two days.

Rail positions measured by Allen's HiRail were compared against 26 centerline targets previously surveyed using RTK GPS. The results for cross-track and vertical error are referenced in table~\ref{tab:Allen}. NDGPS correctors were available during 80\% of the 120 mile traverse of Norfolk Southern mainline track. Two Post-Processed Kinematic (PPK) positions were divided into two categories by distance to the CORS. A first category of observations was processed against a CORS at under 65 miles distant from the survey vehicle, with a second category of between 65 and 130 miles distant from the survey vehicle.

% Table of results: Allen
\begin{table}[ht]
\begin{center}
	\caption{Track Measurement Results~\citep{2006AllenAssetMap}}\label{tab:Allen}
	\begin{tabular}{ lcc }
	\toprule
	Measurement &  Cross-Track & Vertical\\
	System &  Error(ft/95\%) & Accuracy(ft/95\%)\\
	\midrule
	NDGPS & 5.2418 & 13.5308\\
	%\midrule
	PPK $<$ 65 & 1.5758 & 4.4049 \\
	%\midrule
	PPK 65$-$130  & 2.9084 & 10.2528\\
	\bottomrule
	\end{tabular}
\end{center}
\end{table}

The cross-track differences between the previously surveyed track locations and those measured by HiRail summarized in table \ref{tab:Allen}. The results indicate that the PPK accuracies are insufficient for determining track alinement from NDGPS or observations post-processed with observations from individual CORS.

The cross-track accuracy determined from the experimental reaffirm the FRA report on NDGPS that track occupancy ``When viewed as a two dimensional area problem, it is unlikely that any economically feasible [GPS] system could achieve this accuracy to the required $0.9_5$ probability~\citep[pp.6-7]{1995FRADiffe}.''

Other track asset mapping systems exist for determining the location and type of of assets held by railroads. The Union Pacific Railroad has developed and markets a HiRail-based measurement vehicle built on a SUV chassis and referred to as the Precision Measurement Vehicle (PMV). The PMV is used to provide location and description of all assets that can be measured from the railway. While occupying active mainline track, PMV operators use several measurement technologies to determine asset location. Four independent encoded wheels provide linear referenced track position inputs by by accumulating the slope distance between wayside monuments. A differential GPS receiver provides mapping-grade absolute position, while a fiber optic gyroscope (OG) is used to measure grade. The OG also serves to dampen the elevation observations from the DGPS. A video interface provides the operator with a view through optical distance measuring instruments. A video recorder provides a record for milepost tracking and a survey log. Comparative positions generated by the PMV were not available for examination~\citep{2008pmv}.

Glaus details the development of a lightweight multi-sensor track surveying platform. The 99-pound (45 kg) hand-propelled device, nicknamed the \emph{Swiss Trolley}, tested several sensors and the development of a rigorous mathematical model for calculating kinematic track location. Close tolerance rail alinements are required for high speed passenger rail service. The \emph{Swiss Trolley} fills the need for precision track surveying by demonstrated the ability to determine absolute track axis position to a precision of several millimeters. The Swiss Trolley sensor suite is summarized in table \ref{tab:SwissTrolley}.

% Table of instruments: Glaus
\begin{table}[ht]
\begin{center}
	\caption{Swiss Trolley Instruments~\citep{2006glaus}}\label{tab:SwissTrolley}
	\begin{tabular}{ lll }
	\toprule
	Measurement & Sensor & Range/Resolution\\
	\midrule
	Absolute position & RTK GPS Total Station & 1 mm [sic]\\
	Absolute/relative position & Tracking Total Station & 1 mm\\
	Linear distance & Odometer & 0.08 mm\\
	Cross level/Grade & Inclinometer & $\pm$15$^{\circ}$\\
	Asset location & Laser scanner & 180$^{\circ}$\\
	& & 1 mm @ 32 m\\
	& & 10 mm @ 80 m\\
	Gauge & Angular transducer/rail contact & 0-45$^{\circ}$ \\
	\bottomrule
	\end{tabular}
\end{center}
\end{table}

Analog sensors on the \emph{Swiss Trolley} are linked to a control and data acquisition computer by means of an analog to digital multiplexor. Sensors are synchronized with 1 pps timing pulses generated by the GPS receiver. The sensor suite provides inputs to the model to calculate track axis, grade, cross level, and gauge. Points of concern are raised by the author in dealing with thermal and electrical noise on the analog-to-digital (ADC) converter inputs from a variety of sensors. Electromagnetic compatibility interaction between instruments was addressed by reducing the length and attention to the orientation of cables between sensors and ADCs. Two fluid-damped pendulum sensors provide cross-level and grade inputs. These inclinometers are subject to errors from nuisance vibrations, temperature, and collimation (axis alignment) error. Thermal instability errors were reduced by installing the sensors in an instrument oven to maintain a constant temperature regardless of ambient conditions. Grade and cross-level sensor vibration is modeled as a pendulum and applied to track position corrections. Significant is Glaus's method of integrating auxiliary instruments using a Kalman filtering techniques to produces spatial accuracies in the range of several millimeters in a complex dynamic application.

The \emph{Swiss Trolley} is capable of producing exceptional track position accuracies, but the hand propelled sensor suit is limited to a maximum speed of 3.3 mph. Survey speeds are intentionally kept at a minimal to reduce sensor synchronization uncertainty for the platform. The reduced rate also aids in providing an accurate absolute time tag for kinematic data collection~\citep{2006glaus}. The tight instrument integration and slow survey speed make this approach impractical for track inspection survey over wide areas.

\section{Real Time Kinematic Technology}
Augmentation technologies such as Real Time Kinematic (RTK), provide the capability of centimeter-level GPS positioning in real time while the receiver is in motion. RTK technology was assessed in the USDOT Final Report on NDGPS as poorly suited for positive train control. The reports states
``As railroads continue to deploy CBTC\footnote{Communications Based Train Control} and similar GPS-based train management and asset management systems, they must survey the railroad in GPS coordinates. Railroads cover too much territory to practically employ mobile survey-grade reference stations for these surveys\ldots~\citep[pp.12]{2008USDoT_NDGPS}.'' Transmitting RTK correctors to mobile users was characterized as requiring
\begin{quotation}``\ldots their own wireless link between the reference station and the user receiver, which is typically limited to line-of-sight. If the user moves out of range (radio range or line of sight) of the reference station, the reference station must be re-positioned, and the user must again wait for the reference station to achieve ``lock'' with the GPS satellites required for high accuracy~\citep{2008USDoT_NDGPS}.''\end{quotation}
The USDOTs summary disclaims the use of RTK augmentation as ``not usable for general transportation applications''\citep[ES-7]{2008USDoT_NDGPS}.

This research disputes the 2008 USDOT assessment by failing to acknowledge the convergence of several technological factors enabling survey-grade absolute position measurement over wide areas. Demand for high quality positioning from satellite systems has lead to a growth in the availability of public and private alternatives to mapping-grade federal augmentation services. A growing number of state transportation and geodetic survey departments are building their own GPS/GNSS reference networks, providing survey-grade augmentation at no or nominal cost~\cite{ODOTvrs,MDOTvrs,NCvrs,KYCORS}. Unlike federal systems, state and private systems are not limited to providing augmentation for only GPS SVs. Private investment in networked reference systems provides a market opportunity for firms to profit from the need for survey quality GNSS measurement. Delivering real time correctors to a mobile receiver is enabled by the increased capacity to transmit data by a number of means across wide areas. Wireless data transmission comes at a reasonable price and with data transmission rates several orders of magnitude greater than federal GBAS.

CORS networked to deliver observations as real time inputs to a virtual reference server adds the capability to continuously estimate the distortions of SV carrier phase observables while suppling correctors securely to mobile receivers.  Mobile receivers use the VRS server supplied corrections to almost instantaneously refine local observations to within several centimeters. Mobile RTK users can expect to achieve position accuracies of 1-2 centimeters horizontal and 2-3 centimeters vertical across an entire VRS network with 95\% confidence.

The proposed research seeks to bridge the cap between mapping-grade track asset surveys and complex track survey systems. This study will examine the ability of real time kinematic augmentation to enable selected railroad applications not achievable through current federal augmentation services.

\section{Manifest Freight and Hump Yard Efficiency}
Car load freight traffic requires a systematic method for handling the distribution of car destinations, the return of empties, and redirecting cars to their originating industry. On-time delivery in carload service requires minimizing delay during a series of independent car handling events. Beshers notes that each transit through a terminal decreases the probability of an on-time delivery, and provides a statistical determination of overall freight service reliability as the product of the probability-of-delay each time a car is handled along the way to its destination~\citep{2004beshers}. Moorman points to the difficulty rail freight carriers have in achieving acceptable carload service levels in retaining market share~\citep{2006moorman}.

Automated classification yards\footnote{Commonly referred to as hump yards.}, are facilities engineered to continuously process incoming freight cars into outbound trains. Car processing in a hump yard uses the force of gravity to propel cars through a complex of tracks to the intended destination in the yard.

Beshers point to the degradation of car-load service quality and movement away from boxcars to inter-model freight as factors that have resulted in a clear trend towards rail company closure or repurposing of hump yards rather than investing in new facilities~\citep{2004beshers}. Several older hump facilities remain in use but have been repurposed as flat switching yards, as in Russell, KY, Dewitt, NY, and Enola, PA. Others have been converted into intermodal facilities, such as Norfolk Southern Atlanta, GA and Rutherford, PA yards~\citep{2002HumpTrains}.

Dr. Edwin Kraft in a patent claiming to increase yard throughput through a multi-stage sorting algorithm, points out that surviving hump yards operate at close to maximum throughput and operate under a state of constant congestion, to the point that they often cannot accept newly arriving trains. In these circumstances cars are parked on main line track and wait to be processed. Hump yard congestion affects rail service reliability across the network, which in turn contributes to further loss of rail traffic to the trucking industry~\citep{KraftPaten00}.

A hump yard's profile deviation and settlement away from the design grade can be attributed to the effects of car loading forces and weather. Szwilski and Kerchof's observations that yard delays due to grade irregularities are difficult to identify due to the limited yard profile data available to the hump yard engineer~\citep{2005szwilski_kerchof}. When interviewed, Barnes echoed Szwilski and Kerchof in noting that the limited availability of profile data is due primarily to the difficulty in conducting a survey to profile track~\citep{2007barnes}.

Szwilski and Kerchof estimated a differential level survey across a 72-track hump yard would take 4 to 6 weeks of field work by a three-person survey crew. Differential level point density is typically measured on 100' stations, resulting in the observation of approximately 3,000 points. The 480 to 720 man-hours of exposure to rolling stock in an active yard evidences the need to insure the party's safety by closing groups of tracks to production activity. Extended track closures require rerouting railcars away from the yard to prevent yard congestion. The associated cost to reroute railcars is difficult to estimate~\citep{2005szwilski_kerchof}. Safety consideration for the survey party require the yard manager to dedicate specifically trained workers from the yard's labor pool to act as a safety escort. The specter of six continuous weeks of negative productivity from a yard track survey limit a manager's tolerance for obtaining bowl track profiles~\citep{2007barnes}.

The the hump end of an automatic classification yard controlls effect on the motion of a railcar through the yard. Hump end yard operations are described in \href{http://www.youtube.com/watch?v=ndryMwF41Kk}{this video} and provide an graphic understanding of railcar pacing, car variety, and braking. The video follows several railcars from release by the pin puller, through the main, intermediate, and group retarders, passing lead and group switches into the bowl to couple at a speed no greater than 4 mph. The requirement personal protective equipment in or near a hump yard is obvious from the retarder squeal in the audio track.

Anecdotal inference of yard infrastructure problems is the usual mechanism by which grade renewal is scheduled. The reaction of a car's motion through the yard can be observed by paying attention to the suspension. Modern hump yard control systems, such as the ProYard and ProYardII systems, are able to count car transits through the yard network as cars pass wheel detectors (magnetic proximity switches) linked to  PLCs\footnote{Programable Logic Controllers} programmed with timing logic. Misroutes and car stalls recorded by the hump control system are used as metrics in determining yard performance.

This \href{http://www.youtube.com/watch?v=ygl_X9RER70}{time lapse video} shows a series of stalled flat railcars. The opening frames show a flatcar moving away from the hump striking a group of stalled flatcars then rolling backwards, with the car's final rest position blocking the group switch. The blocked switch prevents any further railcars from classification into the blocked group. The stall then effectively shuts down the yard once the next car sequenced for the blocked group reaches the pin puller. The 21 minutes represented in the time-lapse video from stall to the trim locomotive kicking the cars into the alley is indicative of the type of delays affecting the on-time quality of car load service.

\section{Mainline Track Horizontal Alinement}Mainline railways are periodically inspected with specialized ``track geometry cars'' for compliance with FRA mandated track alinement criteria or to meet more stringent rail company specifications~\citep{49CFR213D}~\citep{2009bright.rtrack}. Alinement data recorded by a track geometry cars produce positions with odometers for track location within a linear reference system relative to wayside monuments. Wayside references commonly referred to as mile posts are typically a steel signpost inserted into the ballast beyond the field-site foul point. These markers are less than permanent, and are subject to destruction or displacement from routine track maintenance and vandalism. Resetting mile posts is a best guess effort. Track geometry car measurements are referenced to these moving marks, increasing the difficulty in comparing track geometry over time~\citep{2009vanPelt}.

The linear measurement system of a track geometry car produces accurate relative positions, but is  unable to place the track location accurately within a global reference. Track geometry vehicles are limited in their inspection frequency, and are generally made across a given track segment quarterly~\citep{2009bright.rtrack}.

With the successful completion of the research, absolute track position determined by RTK GNSS will be used to determine relative horizontal track alignment.

\section{Track Occupancy}The FRA has established that any location determination system suitable for supporting positive train control must establish track occupancy with a high degree of certainty. In a given parallel, multitrack segment, an LDS must be able to determine which track a given train is on with almost absolute certainty. The FRA therefore requires an LDS to demonstrate track position that assumes a minimum track separation (center to center spacing) of 11.5 feet with 99.999\% confidence~\citep[4-5]{1995FRADiffe}.

\section{Literature Review Summary}

Recent references to determining track position using satellite navigation systems to measure absolute track position in real time indicate a gap exists between the use of dedicated survey platforms. Allen showed that federally provided NDGPS augmentation closely coupled with an IMU does not provide the necessary precision for determining track alinement or occupancy, while Glauss's sophisticated multi-sensor array is incapable of deployment in wide area track inspection tasks. Review of patent claims indicates addition work exists in industry that seeks to integrate federally provided augmentation with IMUs, but these intellectual property claims are not yet evident in commercial products.

RTK infrastructure receivers networked to a VRS server form a relatively new technology that enable absolute position measurement to within a centimeter over wide areas. It is this unexplored capability that forms the basis for the proposed research.

\section{Contribution of the Study}
By proving the value of RTK GNSS, this study will contribute methods by which track positions in an absolute reference frame can be performed with COTS instrumentation and common rail vehicles to:
\begin{enumerate}
\item Evaluate the ability of RTK augmentation to produce an accurate vertical profile performed by locomotive to measure bowl track profiles safely, rapidly, and accurately without disruption to yard production.

\item Evaluate the ability of RTK augmentation to accurately produce relative horizontal track alignment referenced to track mile posts in degree of curvature from absolute track position. The research will evaluate whether the use of RTK augmentation on a common track vehicle can accurately quantify horizontal alinement over a wide expance of miles of mainline track in a single day.

\item Act as a component of a location determination system suitable for supporting positive train control by establishing track occupancy with a high degree of certainty. In a given parallel, multitrack segment, an LDS must be able to determine which track a given train is on with near absolute certainty. The FRA therefore requires an LDS to demonstrate track position that assumes a minimum track separation (center to center spacing) of 11.5 feet with 99.999\% confidence~\citep[4-5]{1995FRADiffe}.
\end{enumerate}